% !TEX root = 0_main.tex
\chapter{Garbling and Evaluating Sequential Circuits}\label{chap:sequen}
Sequential circuits can be used as a very compact circuit description.
In the following section we first describe the concept of sequential circuits (\sect{sec:sqcirc}) using an example and then explain the modifications required to garble/evaluate them.

\section{Sequential Circuits}\label{sec:sqcirc}

\begin{figure}[ht]
    \centering
    \begin{subfigure}[t]{0.35\textwidth}
        \includegraphics[width=\textwidth]{combinational-crop.pdf}
        \caption{Combinational circuit}\label{fig:combinational}
    \end{subfigure}
    \begin{subfigure}[t]{0.30\textwidth}
        \includegraphics[width=\textwidth]{sequential-crop.pdf}
        \caption{Sequential circuit}\label{fig:sequential}
    \end{subfigure}\\
\vspace{10pt}
    \caption{(a) Combinational circuit where outputs are functions of only inputs.
(b) Sequential circuit where outputs are functions of inputs and present states.}
\end{figure}

Yao's GC algorithm allows secure evaluation of a Boolean circuit, i.e., an acyclic graph of binary gates (e.g., AND, OR, XOR, etc.).
In digital circuit theory, such a circuit is called \emph{combinational circuit} and defined as a memory-less circuit in which outputs are functions only of inputs, see \fig{fig:combinational}.

Another class of circuits in digital circuit theory are \emph{sequential circuits} in which unlike in the combinational case, circuit outputs are functions of both inputs and circuit \emph{states}.
Circuit states are kept in memory elements such as Flip Flops (FF).
The states can change at the end of each \emph{clock cycle}\footnote{The clock signal oscillates between a low and a high state and its (rising) edge is typically utilized to coordinate the memory updates.}.

As seen in \fig{fig:sequential}, a sequential circuit can be represented as an ensemble of a combinational circuit and feedback loops with memory elements.
At each clock cycle, circuit inputs as well as the present states are fed to the combinational part.
Then, it generates the outputs and next states which will be stored in the memory elements for the next cycle.
The initial value of the memory elements are either a known constant value ($0$ or $1$) or determined by an initial input value\footnote{In digital hardware, FF initialization is usually done by \emph{reset} or \emph{set} signals.
In \sys{}, we use a new signal for FF that determines the initial value.
It can be connected to a constant value or input wire.}.

\begin{figure}[h!]
    \centering
    \includegraphics[width=0.7\textwidth]{adder-crop.pdf}
    \caption{Combinational and sequential design of a 4-bit Adder.
  (a) HA circuit.
  (b) FA circuit.
  (c) Combinational 4-bit Adder using 1 HA and 3 FAs.
  (d) Sequential 4-bit Adder using one FA.}\label{fig:combSeq}
\end{figure}

\fig{fig:combSeq} demonstrates an example of a combinational and a sequential implementation for a 4-bit Adder with inputs $x = \overline{x_3x_2x_1x_0}$ and $y = \overline{y_3y_2y_1y_0}$, producing sum $s = \overline{s_3s_2s_1s_0}$.
\fig{fig:combSeq}a and \ref{fig:combSeq}b show the internal combinational circuit of a half Adder (HA) and a full Adder (FA) respectively.
In \fig{fig:combSeq}c a combinational Adder is built by cascading 3 FAs and one HA.
\fig{fig:combSeq}d represents a sequential implementation of a 4-bit Adder which uses one FA and a one bit FF to save the carry bit from the previous cycle.
The circuit should be evaluated in 4 cycles.
At the first cycle the carry bit is $z_0=0$.
Note that, in the combinational circuit we use three FAs and one HA whereas in the sequential circuit, we have to use one FA for 4 sequential cycles.
This asymmetry in the loop of Addition function introduces a very small \emph{overhead} in GC computation and communication time as an HA circuit has fewer gates compared to a FA circuit.

However, the total number of gates for representing the function is reduced approximately by a factor of 4 when using a sequential circuit (one FA for sequential compared with three FA and one HA for combinational).
This helps to limit the memory footprint for garbling and evaluation required for storing circuit description and wire tokens ($k$-bit per wire, see \sect{subsec:preli_GC}).
In a sequential circuit, the number of tokens that need to be stored in memory at any moment is proportional to the number of gates in the circuit.
The wire tokens are simply over-written at each sequential cycle.
Only tokens corresponding to FFs are kept for the next cycle.

Nearly all commercial circuits used in digital hardware are designed in sequential format.
There are multiple reasons for preferring sequential circuit description over combinational including the reduction in complexity, area, power, and cost, as well as natural mapping of finite state machine control functions into a sequential format.
Some of these reasons also provide a rationale for sequential description of a function in GC, including: (i) reduction in size and memory footprint that is achieved by introducing the state elements and feedback loop from output to input; (ii) removing the need to perform costly compile-time/runtime loop unrolling by embracing loops within the sequential feedback loop; (iii) providing a new degree of freedom for folding by the placement of memory elements in the long combinational paths--the placement can be done in accordance with the user's objective.

\begin{figure}[t]
    \centering
	\includegraphics[width=0.8\textwidth]{sequential-open-crop.pdf}
	\caption{Functionally equivalent unrolled sequential circuit corresponding to \fig{fig:sequential}.}
	\label{fig:open-sequential}
\end{figure}

During the evaluation of a sequential circuit, the combinational block is evaluated $c$ times where $c$ is the number of sequential cycles that the circuit operates.
We can visualize this process as the unrolled combinational representation of the sequential circuit as shown in \fig{fig:open-sequential}.
The inputs/outputs of the unrolled circuit are the inputs/outputs of the combinational block in all the cycles.
The present states at each cycle $\textrm{cid}$, where $0 \le \textrm{cid} < c$, are equal to the next states at the previous cycle ($\textrm{cid}-1$).
The present states at $\textrm{cid}=0$ are equal to the input initial value.

During generation and evaluation of the garbled circuit, it must be ensured that the encryption tweaks $T$ (see \sect{subsec:preli_GC}) for each gate is unique because otherwise security is broken \cite[Sect. 3.4]{HS13}.
In \sys{}, to ensure the uniqueness property, we set tweak $T$ for each gate to be the concatenation of the cycle index (cid) and the unique gate identifiers (gid) in the combinational part of the sequential circuit, i.e., $T = \textrm{cid} || \textrm{gid}$.
\footnote{An alternative method would be to use a monotonic counter in the circuit generation/evaluation routines which is increased by one for each gate.}
As in previous work, security and correctness of the GC garbling/evaluation follow from the uniqueness of the tweak $T$ and the existing proofs of security and correctness, see \cite{LP09,bellare2013efficient}.

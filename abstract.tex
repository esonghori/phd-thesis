% !TEX root = 0_main.tex
\thispagestyle{empty}
\begin{abstract}
Privacy-preserving computation is a standing challenge central to several modern-world applications which require computing on sensitive data.
\acrfull{sfe} refers to provably secure techniques aiming to address this challenge by enabling multiple parties to jointly compute an arbitrary function on their private inputs.
The most promising two-party \acrshort{sfe} method is called the \acrfull{gc} protocol introduced by Andrew C. Yao.
The protocol is built upon representing the function as a Boolean circuit and encrypting/communicating at logic gate level.
Despite several key progresses in \acrshort{gc}, efficiency, scalability and ease-of-use of the available methods are limited by the naive circuit representation as a directed acyclic graph, ad-hoc logic optimizations, and customized compilers.


In this thesis, we proposed a holistic solution to enhance the efficiency, scalability, and simplicity of the \acrshort{gc} protocol.
Our approach has three main pillars to address these major challenges: \acrlong{gc} synthesis, sequential \acrlong{gc} and garbled processor.
The \acrlong{gc} synthesis is a novel automated methodology based on logic synthesis techniques for generating optimized Boolean circuits for the \acrshort{gc} protocol.
Moreover by introduction of sequential \acrlong{gc}, we achieve an unprecedented level of compactness and scalability by using a sequential circuit description.
We combine \acrlong{gc} synthesis, sequential \acrlong{gc} in an open-source  framework called \gls{tinygarble}.
The preliminary implementation of benchmark functions using \gls{tinygarble} demonstrates a high degree of memory-foot-print compactness as well as improvement in overall efficiency compared to results of existing automated tools.

Our sequential description also enables us, for the first time, to design and realize a garbled processor to reduce the problem of private function evaluation to a conventional \acrshort{sfe} problem.
In addition, the garbled processor allows to develop applications in high-level languages and securely evaluate them which eliminates the need for Boolean circuit generation.
We adopt the garbled processor for conventional \acrshort{sfe} problems where the function do not need to kept private.

We present \frwk{}, a novel secure function evaluation framework based on Yao's Garbled Circuit (GC) protocol and ARM processor.
It allows users to develop privacy-preserving applications using high-level programming languages with comparable efficiency to the best prior results achieved using conventional logic synthesis tools and hardware description language.
\frwk{} applications are developed in a high-level language (e.g., C) and compiled using standard verified ARM compilers (e.g., gcc-arm).
In the \frwk{} framework, the Boolean circuit is that of an ARM processor to which the compiled binary code of the function is input as a non-private instruction code.
The main enabler to make this construction practical and efficient is the introduction of \sys{}, a new algorithm that omits the communication and encryption cost of a Boolean gate when its output is independent of the private data.
Benchmark evaluations demonstrate efficiency and usability of \frwk{} compared with prior art in high level GC compilation.


\end{abstract}

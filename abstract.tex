% !TEX root = 0_main.tex
\thispagestyle{empty}
\begin{abstract}
Privacy-preserving computation is a standing challenge central to several modern-world applications which require computing on sensitive data. \acrfull{sfe} refers to provably secure techniques aiming to address this challenge by enabling multiple parties to jointly compute an arbitrary function on their private inputs. The most promising two-party \acrshort{sfe} method is called the \acrfull{gc} Protocol introduced by Andrew Yao. The protocol is built upon representing the function as a Boolean circuit and encrypting/communicating at logic gate level. Despite several key progresses in \acrshort{gc}, efficiency and scalability of the available methods are limited by the naive circuit representation as a directed acyclic graph, and ad-hoc logic optimizations. In this work, we introduce \gls{tinygarble}, a novel automated methodology based on logic synthesis techniques for generating optimized Boolean circuits for \acrshort{gc} protocol. Moreover, \gls{tinygarble} achieves an unprecedented level of compactness and scalability by using a sequential circuit description. The preliminary implementation of benchmark functions using \gls{tinygarble} demonstrates a high degree of memory-foot-print compactness as well as improvement in overall efficiency compared to results of existing automated tools. Our sequential description also enables us, for the first time, to design and realize a garbled processor to reduce the problem of private function evaluation to a conventional \acrshort{sfe} problem. In addition, the garbled processor allows to develop applications in high-level languages and securely evaluate them which eliminates the need for Boolean circuit generation.
\end{abstract}

\chapter{Garbled Circuit Library}\label{sec:library}
In this section, we introduce \gls{tinygarble} library that consists of \acrshort{gc} optimized circuits for complex mathematical/logical operations that can be used as a building blocks practical applications.
The SSCD and netlist files of the circuits in the library can be found in \gls{tinygarble} repository\footnote{\url{https://github.com/esonghori/\gls{tinygarble}/blob/master/scd/netlists/v.tar.gz}}.


\section{Division and Remainder}
The \gls{tinygarble} library includes the circuit for integer division which takes two numbers as inputs and outputs quotient and remainder (modulus).
The library includes the circuits for 16-, 32-, and 64-bit integers.
Two additional circuits are also included in the library that output only quotient and remainder respectively and have fewer number of gates compared to the original circuit.

\section{Floating Point Operations}
The \gls{tinygarble} library includes an extensive set of operations for both IEEE-754 single- and double-precision floating-point numbers.
It supports addition (fp\_add), subtraction (fp\_sub), division (fp\_div), and multiplication (fp\_mult) of two inputs.
A comparison circuit (fp\_cmp) is also included that outputs three bits: less than, greater than, equal signals.
The natural exponentiation circuit (fp\_exp) receives a floating-point input $a$ and computes $e^a$.
The logarithm circuit (fp\_log2) calculates logarithm of the input in base 2.
Other floating-point operations include square (fp\_square), and square-root (fp\_sqrt).

\section{Encoder}
Encoder circuit converts a one-hot representation into a binary one.
One-hot representation has only one active bit (usually equal one).
Number $i$ is represented in one-hot by activating the $i^{\text{th}}$ bit.
Thus, the encoder circuit outputs the index of the active input bit of the one-hot representation.
Therefore, for $n$-bit one-hot input, the output $o$ is $\log(n)$ bit wide.
For example, if the $n=16$ and the $5^{\text{th}}$ bit is one, the output should be $o=0101$ (4-bit wide).
If none of the inputs is one, the encoder outputs zero.

We implement the encoder operation using a recursive structure.
An encoder for $n$-bit input is implemented using two smaller encoder for $n/2$-bit input.
the first half of the input ($0$ to $n/2-1$) is given to the first small encoder and the rest ($n/2$ to $n-1$) is given to the second one.
One of the these two small encoder that receives the inactive part of input outputs zero and the other one outputs the binary representation in its half.
Depending on which half was active, a bit is attached as the Most Significant Bit (MSB) to the final output.
In other words, the MSB of $o$ is set to one if the output of the second encoder is nonzero and otherwise zero.

\section{Argmax}
Given an array of numbers, Argmax circuit outputs the largest number along with its index in the array.
Current implementation supports integers but it can easily be extended to support any type of inputs with appropriate comparison block.
The size of the array ($n$) and the number of bits ($b$) for each number are parameters of the circuit and can be set before compilation.
The combinational circuit of Argmax has $n-1$ comparison blocks, $n-1$ two-to-one $b$-bit-wide MUXs, and $n-1$ two-to-one $\log_2{n}$-bit-wide MUXs.
Its sequential circuit has one comparison blocks, and two MUXs for selecting the index and the largest number and two registers to store them.
The sequential circuit has to be evaluated for $n-1$ sequential cycles.

\section{K-Nearest Neighbors Search}
This module outputs the K-Nearest Neighbors (KNN) of a given point from set of points.
Each point is described as a binary vector.
The closeness metric can be any arbitrary measurement (here: Hamming distance) and can be implemented separately as well for different similarity metrics.
The KNN module is implemented as a sequential circuit which takes one point from the dataset at each clock cycle.
The K nearest points are stored in the registers (Flip-Flops) in a sorted fashion.
That is, register 1 stores the closest point to the target point, register 2 stores the second closest point and so on.
These registers are initialized by maximum number possible (all ones in the case of unsigned integers).
At each clock cycle, the distance form the new input point and the target point is computed.
The rest of the circuit acts as a min priority queue where the priority of each item is its distance from the target point.
For example, when K=5 and the correct position of the new point is at position 3, the value of position 3 is stored at position 4 and the value of position 4 will be stored at position 5.
After that the new value is stored at position 3.
The circuit includes 1 Hamming distance computation, $2\times K-1$ Comparison blocks, and $4\times K-2$ two-to-one MUXs.
The sequential circuit has to be evaluated for $n$ times where $n$ is the total number of points in the dataset.
The final output of the circuit are the K nearest neighbors which are stored in the registers.

\section{CORDIC}
COordinate Rotation DIgital Computer (CORDIC) circuit computes hyperbolic and trigonometric functions.
It is an iterative algorithm that improves the accuracy of the result by (typically) one bit at each iteration.
We have implemented CORIDC as a sequential circuit which performs one iteration at each sequential cycle.
It includes a lookup table, shift, addition, and subtraction operations.
CORDIC circuit takes three inputs ($x_0$, $y_0$, and $z_0$) and outputs three values ($x_n$, $y_n$, and $z_n$).
The subscript $n$ denotes the final outputs after running $n$ iterations CORDIC.
All of the inputs, outputs, and intermediate results are in fixed-point representation.
The number of integer bits and fractional bits are parameters that can be set before compilation.

CORIDC has three operation modes:
(i) Circular, (ii) hyperbolic, and (iii) linear.
The circular mode can rotate an arbitrary vector by a given angle.
With specifically chosen inputs ($x_0=1$, $y_0=0$, and $z_0=\theta$) in circular mode, the circuit computes the trigonometric functions ($x_n=\cos(\theta)$, $y_n=\sin(\theta)$, and $z_n=0$).
The circuit computes two different functions without facing any more overhead compared to computing only one of them.
Similarly, the hyperbolic mode can compute hyperbolic functions.
Please note that in the hyperbolic mode iterations $3\times i+1$ need to be computed twice.
The linear mode can compute the multiplication of the inputs ($x_n=x_0, y_n=y_0z_0, z_n=0$).
By leveraging the ability of CORDIC, number of other non-linear functions can be computed as well.
For example,
$\tan(\theta)=\frac{\sin(\theta)}{\cos(\theta)}$,
$\tanh(\theta)=\frac{\sinh(\theta)}{\cosh(\theta)}$,
$\exp(x)=\sinh (x) + \cosh (x)$, and
$Sigmoid(\theta)=\frac{1}{1+\cosh(\theta)-\sinh(\theta)}$.

\section{Evaluation}
\tab{table:div} reports the result of integer devision and reminder functions for various bit-widths.
Div\_rem function calculates both the reminder and division in the same circuit.
A few of the previous work reported the number of non-XOR gates for 32-bit integer division only: 1,437 \cite{mood2016frigate}, 1,210 \cite{zahur2015obliv}, and 2,236 \cite{liu2015oblivm}.
As shown in the table, the number of non-XOR gates for our 32-bit division is 599.
It means 2 to 3.7 times improvement compared to the previous custom compilers.

\begin{table}[]
\center
\caption{The result of integer division and reminder functions.}\label{table:div}
\begin{tabular}{l||r||r|r}
	Function                  & \multicolumn{1}{c|}{Bit-width} & \multicolumn{1}{c|}{Non-XOR} & \multicolumn{1}{c}{Total gates} \\ \hline \hline
\multirow{3}{*}{Reminder} & 16        & 186     & 500         \\
                          & 32        & 631     & 1,777       \\
                          & 64        & 2,290   & 6,636       \\ \hline \hline
\multirow{3}{*}{Division} & 16        & 170     & 452         \\
                          & 32        & 599     & 1,681       \\
                          & 64        & 2,226   & 6,444       \\ \hline \hline
\multirow{3}{*}{Div\_rem} & 16        & 186     & 501         \\
                          & 32        & 631     & 1,778       \\
                          & 64        & 2,290   & 6,637
\end{tabular}
\end{table}

\tab{table:float} illustrates the number of non-XOR and total gates for the floating point operations.
The results of both single precision (float) and double precision (double) are reported in the table.
Pullonen et al, is the only previous work that reported the result for the similar floating-point operations \cite{pullonen2015combining}.
They used the custom compiler of CBMC-\acrshort{gc} \cite{franz2014cbmc} to compile software implementations of these floating-point operations.
As can be seen in the table, the circuits in \gls{tinygarble} library outperforms the implementation in \cite{pullonen2015combining}.

\begin{table}[]
\center
\caption{The result of floating-point functions.}\label{table:float}
\begin{tabular}{l|l||rr||rr||rr}
\multirow{2}{*}{Function}   & \multicolumn{1}{c||}{\multirow{2}{*}{Precision}} & \multicolumn{2}{c||}{\begin{tabular}[c]{@{}c@{}}Previous work\\\cite{pullonen2015combining}\end{tabular}}                             & \multicolumn{2}{c||}{\begin{tabular}[c]{@{}c@{}}\gls{tinygarble} \\Library\end{tabular}}                                & \multicolumn{2}{c}{Comparison}                   \\ \cline{3-8}
                            & \multicolumn{1}{c||}{}                           & \multicolumn{1}{c}{Non-XOR} & \multicolumn{1}{c||}{Total gates} & \multicolumn{1}{c}{Non-XOR} & \multicolumn{1}{c||}{Total gates} & \multicolumn{1}{c}{GTD} & \multicolumn{1}{c}{MFE} \\ \hline \hline
\multirow{2}{*}{fp\_add}    & Single                                          & 5,671                       & 7,052                            & 936                         & 1,562                            & -83.5\%                 & 4.51                    \\
                            & Double                                          & 13,129                      & 15,882                           & 2,030                       & 3,533                            & -84.5\%                 & 4.50                    \\ \hline \hline
\multirow{2}{*}{fp\_sub}    & Single                                          & 5,671                       & 7,052                            & 936                         & 1,562                            & -83.5\%	                & 4.51                    \\
                            & Double                                          & 13,129                      & 15,882                           & 2,032                       & 3,537                            & -84.5\%                 & 4.49                    \\ \hline
\multirow{2}{*}{fp\_mult}   & Single                                          & 5,138                       & 7,701                            & 3,554                       & 4,839                            & -30.8\%                 & 1.59                    \\
                            & Double                                          & 13,104                      & 25,276                           & 17,019                      & 23,484                           & 29.9\%                  & 1.08                    \\ \hline \hline
\multirow{2}{*}{fp\_div}    & Single                                          & 12,851                      & 21,384                           & 3,810                       & 6,275                            & -70.4\%                 & 3.41                    \\
                            & Double                                          & 36,133                      & 73,684                           & 17,585                      & 29,130                           & -51.3\%                 & 2.53                    \\ \hline \hline
\multirow{2}{*}{fp\_cmp}    & Single                                          & -                           & -                                & 213                         & 239                              & -                       & -                       \\
                            & Double                                          & -                           & -                                & 435                         & 463                              & -                       & -                       \\ \hline \hline
\multirow{2}{*}{fp\_exp}    & Single                                          & -                           & -                                & 12,596                      & 16,250                           & -                       & -                       \\
                            & Double                                          & 393,807                     & 579,281                          & -                           & -                                & -                       & -                       \\ \hline \hline
\multirow{2}{*}{fp\_log2}   & Single                                          & -                           & -                                & 13,072                      & 16,727                           & -                       & -                       \\
                            & Double                                          & -                           & -                                & 16,355                      & 23,484                           & -                       & -                       \\ \hline \hline
\multirow{2}{*}{fp\_square} & Single                                          & -                           & -                                & 1,763                       & 2,391                            & -                       & -                       \\
                            & Double                                          & -                           & -                                & 8,461                       & 11,457                           & -                       & -                       \\ \hline \hline
\multirow{2}{*}{fp\_sqrt}   & Single                                          & 35,987                      & 66,003                           & 1,842                       & 3,360                            & -94.9\%                 & 19.64                   \\
                            & Double                                          & 85,975                      & 169,932                          & 8,636                       & 15,801                           & -90.0\%                 & 10.75
\end{tabular}

\end{table}

% !TEX root = 0_main.tex
\chapter{Related Work}\label{chap:related}
In this chapter, we review the related work to this thesis in the literature.
First, we classify the prior art for generating a circuit for Yao's \acrfull{gc} protocol into custom compilers in \sect{sec:related-compiler}, optimized libraries in \sect{sec:related-library}, and garbled processor in \sect{sec:related-processor}.
Next, we review the similar work on logic synthesis for other circuit-based \acrfull{sfe} protocol in \sect{sec:related-logic}.
We then discuss the work on \acrshort{gc} implementations with hardware accelerators in \sect{sec:related-hardware}, and \acrshort{gc} realization on embedded devices in \sect{sec:related-embedded}.
Lastly, we review the work on securing the \acrshort{gc} protocol against a malicious adversary in \sect{sec:related-malicious}.

\section{Compiler for Garbled Circuits}\label{sec:related-compiler}
The idea of designing a custom programming language to describe and efficiently compile functions for secure evaluation dates back to Fairplay, the first practical implementation of the \acrshort{gc} protocol introduced in 2004 \cite{malkhi2004fairplay}.
Fairplay introduced a custom high-level procedural language called \acrfull{sfdl}.
The user has to write their privacy-preserving applications in \acrshort{sfdl} and translate them into Fairplay's circuit description language, \acrfull{shdl}.

Henecka et al. (2010) introduced TASTY compiler \cite{henecka2010tasty} which allows combining garbled circuits and Homomorphic encryption.
In TASTY, the user can develop an \acrshort{sfe} program in a new domain-specific language to be performed on private data.
The compiler then translates the program into a combination of the \acrshort{gc} protocol and Homomorphic encryption to securely evaluate the program.

Mood et al. (2012) introduced a new Pseudo-Assembly Language (PAL) for \acrshort{gc} circuit generation \cite{mood2012memory}.
They designed a new compiler that translates a program in the PAL into Fairplay \acrshort{sfdl} to generate circuits with limited memory budget.

Kreuter et al. (2012) introduced a \acrshort{gc} compiler that for the first time addressed the scalability issue of generating and garbling large circuits under the malicious model~\cite{kreuter2012billion}.
Their compiler can produce circuits consisting of billions of gates, e.g., a 4095x4095-bit Edit Distance circuit with almost six Billion gates.

The first \acrshort{gc} framework supporting a general-purpose language is presented in \cite{holzer2012secure}, which supports \acrshort{ansi}-\gls{c}.
However, it supports only a subset of \acrshort{ansi}-\gls{c} that is not compatible with many crucial primitives and therefore, not compatible with legacy codes.
The main drawback of~\cite{holzer2012secure} is compile-time loop unrolling that makes it unscalable with input size.

Kreuter et al. (2013) introduced \gls{pcf} compiler for the \acrshort{gc} protocol \cite{kreuter2013pcf}.
\gls{pcf} does not unroll loops in the program until the \acrshort{gc} evaluation to reduce the memory overhead.
The loops have to be marked manually in the high-level language.
In contrast to \gls{pcf}, \gls{tinygarble} allows to infer loops automatically and also allows to optimize across multiple sub-circuits.

Franz et al. (2014) introduced CBMC-GC, a new \acrshort{gc} compiler in \cite{franz2014cbmc}.
Their compiler lets users program in \gls{c} language and translate the programs into a Boolean circuit.
The objective of their compiler is to reduce the number of non-XOR gates in the generated Boolean circuits.

Rastogi et al. (2014) proposed WYSTERIA, a new high-level programming language for the \acrshort{gc} protocol in \cite{rastogi2014wysteria}.
WYSTERIA, unlike previous work, allows a combination of local and private computations on the data.
WYSTERIA also introduces a new abstraction for secret shared data that support generic n-party secure computations.

Liu et al. (2015) proposed ObliVM framework for secure computation in \cite{liu2015oblivm}.
They proposed a new domain-specific language that ObliVM compiles into an efficient representation for secure computation.
Most significantly, ObliVM allows users to benefit from the sub-linear cost of oblivious array accesses in the \acrshort{gc} protocol using Circuit \acrfull{oram} scheme \cite{wang2015circuit}.

Zahur et al. (2015) proposed a new compiler for \acrshort{gc} that relies on a standard \gls{c} compiler \cite{zahur2015obliv}.
Their compiler lets users program in normal \gls{c}.
They also make their compiler compatible with Path \acrshort{oram} scheme \cite{stefanov2013path} which reduce the cost of oblivious access to arrays.

Mood et al. (2016) provided a comprehensive study of the state-of-the-art compilers~\cite{mood2012memory,kreuter2012billion,kreuter2013pcf, franz2014cbmc,zahur2015obliv,liu2015oblivm} for performing secure function evaluation using high-level languages in~\cite{mood2016frigate}.
They showed that the majority of such compilers are not thoroughly validated and they reported the observed flaws in six commonly used platforms.
As they discussed in their paper, there are severe limitations for formal verification, and due to its impracticality, they limit their analysis to validation by testing.
This type of testing does not detect all possible flaws in the compilation process.

Mood et al. (2016) also introduced \gls{frigate}, a new \gls{c}-style language for \acrshort{sfe} and the corresponding compiler in the same paper~\cite{mood2016frigate}.
\gls{frigate} supports three different types (\texttt{uint\_t}, \texttt{int\_t}, and \texttt{struct\_t}).
The user can add her types, but it requires a good understanding of the internal structure of the compiler.
Since these three types have a specific bit length, the final computation is not bit-level efficient.
For example for a 9-bit comparison, \gls{frigate} needs to perform the comparison for a given bit length of \texttt{int\_t}.
On the contrary, the \gls{arm2gc} framework eliminates unnecessary gates and evaluates the circuit only up to the number of bits needed.
\gls{frigate} divides the program into different functions and creates the circuit by calling the corresponding functions and as a result prohibits the overall circuit optimization.
In contrast, our \gls{arm} circuit is optimized globally using state-of-the-art hardware synthesis techniques.
Therefore, our overall platform relies on very well-developed and debugged tools that have been used in industry for many years.
Also, if any new update becomes available for these tools, they can effortlessly be incorporated into our framework.

The introduction of a new custom programming language is neither user-friendly nor versatile when compared with a conventional programming language like \gls{c}.
Moreover, the user has to compile her code with a newly designed custom compiler in these works.
As a result, the user cannot benefit from the optimizations provided by general-purpose and standard compilers.
Furthermore, these compilers are less scrutinized and therefore more prone to bugs.
In contrast, the \gls{arm2gc} framework supports any general-purpose \gls{arm} compiler and thus benefit from all the state-of-the-art optimizations, supports legacy codes, and is fully verified.

\section{Libraries for Garbled Circuits}\label{sec:related-library}
Instead of compiling circuits, Huang et al. (2011) proposed the FastGC framework that uses a library-based approach where circuits can be programmed and integrated into high-level applications \cite{huang2011faster}.

Another library-based \acrshort{gc} framework is VMCrypt proposed by Malka et al. (2011) in \cite{malka2011vmcrypt} to address software modularity and scalability issues in the previous \acrshort{gc} frameworks.
VMCrypt provides a modular software architecture in Java programming language that dynamically constructs and deconstructs sub-circuits.
VMCrypt removes sub-circuits from the memory when their computations are done through destructing their objects.

Henecka et al. (2013) extended the FastGC framework in \cite{henecka2013faster} to re-use the same sub-circuits.
They also introduce multi-threaded implementation for \acrfull{ot} and cashing circuit descriptions and network packets.

Demmler et al. (2015) proposed ABY, a library-based framework for secure computation that allows the efficient combination of three secure computation schemes: Arithmetic sharing, Boolean sharing, and Yao’s \acrshort{gc} \cite{demmler2015aby}.
Conversion of the private data between these secure computation schemes is supported using a pre-computed \acrshort{ot}.

Library-based approaches suffer from the fact that user has to decompose the function into sub-circuits manually.
Thus, the user needs to have a thorough understanding of the circuit description.
Whereas, our methods in \gls{tinygarble} and \gls{arm2gc} are automated approaches and provide more abstraction compared to library-based frameworks.

\section{Garbled Processor}\label{sec:related-processor}
Wang et al. (2016) proposed a \acrshort{gc} framework based on \gls{mips} processor in \cite{wang2016secure}.
Their framework accepts a function as a \gls{mips} machine code, which allows the programmer to describe the function in a language of her choice and compile with a standard compiler.
They design a \gls{mips} emulator to execute the code securely.
To avoid emulating a large number of instructions supported by the \gls{mips} processor, they perform a data independent static analysis before execution of the program to build a small instruction bank and \acrfull{alu} circuit tailored for each processor cycle.
In contrast, our \gls{arm2gc} performs this optimization with bit-precision instead of instruction-precision (see \sect{sec:processor-arm}).
Moreover, this is done in the runtime while the circuit remains the same for each cycle.

To solve the problem of the secure conditional branch, Wang et al. (2016) propose to pad \texttt{nop} instruction to parallel branches so that their lengths become equal \cite{wang2016secure}.
This way when the code exits either of the branches, it ends up in the same instruction and the process can continue with less cost.
However, this approach increases the cost for conditional branches.
To mitigate this problem, we propose to use \gls{arm} processor which supports conditional execution and can replace these branches with conditional instructions (see \sect{sec:processor-arm}).
In rare cases where the \gls{arm} compiler fails to replace the conditional branch, we adopted their approach in padding the parallel branches with \texttt{nop} instruction.
Overall, our evaluation shows that \gls{arm2gc} outperforms their MIPS framework, for example by four orders of magnitudes for Hamming distance function, mostly thanks to the \gls{skipgate} algorithm and its bit-precision optimization.

\section{Logic Synthesis for Other SFE Circuit-based Protocols} \label{sec:related-logic}
Inspired by \gls{tinygarble}'s methodology, Demmler et al. (2015) proposed a method to use industrial logic synthesis tools to optimize both size of and depth of Boolean circuits used in secure computation \cite{demmler2015automated}.
The depth of circuit is a crucial factor in the performance of the circuit-based \acrfull{gmw} protocol \cite{goldreich1987play}.
The round complexity of \acrshort{gmw} depends on the number of non-XOR gates in the longest path from input to output.
In \acrshort{gmw} for each layer of non-XOR gates, an \acrshort{ot} needs to executed between the parties.
Demmler et al. (2015) proposed using timing optimization of logic synthesis tools to reduce the depth of the circuit \cite{demmler2015automated}.
They set the area and timing delay of non-XOR gates to a large non-zero value, and those of XOR gates to zero and then force the synthesis tool to reduce the overall delay and area of the circuit.
This approach results in a small and shallow circuit in term of non-XOR gates that in turn, reduces the cost of secure evaluation of the circuit in the \acrshort{gmw} protocol.
The authors also extend their library of the ABY framework \cite{demmler2015aby} with the new optimized circuits including floating-point operations for the \acrshort{gmw} protocol.

\section{GC Implementations with Hardware Accelerators} \label{sec:related-hardware}
The following works provide better performance by implementing garbled circuits in hardware, on \acrfull{gpu}s, or using \acrfull{aes-ni} available in modern \acrshort{cpu}s.
These works can benefit from the compact representation generated by \gls{tinygarble}.

J\"arvinen et al. (2010) \cite{jarvinen2010garbled} proposed a generic hardware architecture for \acrshort{gc}.
They realized two \acrfull{fpga} based prototypes: a system-on-a-programmable-chip with access to a hardware cryptographic accelerator targeting smart cards and smart phones, and a stand-alone hardware implementation targeting \acrfull{asic}.

Several accelerations of \acrshort{gc}s using \acrshort{gpu}s have been proposed recently.
Husted et al. (2013) implemented Yao's \acrshort{gc} using optimizations such as Free XOR, pipelining, and \acrshort{ot} extension \cite{husted2013gpu}.
Pu et al. (2013) realized dynamic programming based on \acrshort{gc} to solve the Edit-Distance and the Smith-Waterman problems \cite{pu2013computing}.
They also used the same optimizations as \cite{husted2013gpu} along with permute-and-encrypt, efficient lookup-table design, and compact circuits \cite{pu2013computing}.
Frederiksen et al. (2013) implemented a secure computation protocol that is secure against malicious adversaries based on a cut-and-choose strategy and an efficient \acrshort{ot} extension for two-party computation on \acrshort{gpu}s \cite{frederiksen2013fast}.

Bellare et al. (2013) propose JustGarble in which they use fixed-key \acrshort{aes} for circuit-garbling \cite{bellare2013efficient}.
They show their implementation using \acrshort{aes-ni} can efficiently garble and execute a circuit far faster than any prior report.

Recently, Fang et al. (2017) proposed a generic implementation of the \acrshort{gc} protocol on \acrshort{fpga} \cite{fang2017secure}.
They propose a coarse-grained architecture that does not require to be reprogrammed for evaluating a new \acrshort{sfe} application.

\section{GC Implementations on Embedded Devices} \label{sec:related-embedded}
Our approach for generating compact circuit representations is also beneficial when performing secure computation on resource-constrained embedded devices such as mobile devices which have a limited amount of main memory.
Huang et al. (2011) proposed secure computation on mobile devices using garbled circuits \cite{huang2011privacy}.
Moreover, the protocol described in \cite{demmler2014ad}, which uses a smart-card installed in the embedded device, can benefit from our more compact circuit representation.
In \cite{carter2016secure, carter2014whitewash}, the mobiles no longer need to process circuits anymore as \acrshort{gc} generation and evaluation is outsourced to cloud servers.

\section{Securing GC against Malicious Adversary}\label{sec:related-malicious}
Yao's \acrshort{gc} protocol discussed in this thesis is secure against an \acrfull{hbc} adversary, also known as a semi-honest or passive adversary (see \sect{ssec:prelim-gc}).
In the \acrshort{hbc} model, the parties follow the protocol, but they wish to learn as much as possible about the other party's private input.


In a malicious model of \acrshort{gc} where parties can deviate from the protocol, Alice can garble a faulty circuit instead of the correct one and send it to Bob.
Bob is not able to distinguish the faulty circuit because it is garbled, i.e., encrypted.
A straightforward method to make the protocol secure against a malicious adversary is to apply Goldreich-Micali-Wigderson (GMW) compiler to the protocol \cite{goldreich1987play}, such that Alice can prove to Bob that she garbled the circuit correctly without revealing the garbling secrets.
However, this approach is not practical due to the use of costly generic zero-knowledge proofs in the GMW compiler \cite{lindell2007efficient}.

Lindell and Pinkas (2007) proposed the first practical and efficient method to secure the \acrshort{gc} protocol against malicious adversaries in \cite{lindell2007efficient}.
They propose to apply a cut-and-choose strategy on the \acrshort{gc} protocol.
Alice garbles the circuit using the \acrshort{hbc} version of the protocol for $s$ times and send the garbled circuits to Bob.
$s$ is a statistical security parameter, and the cheating probability is exponentially small with respect to $s$.
Bob randomly selects half of the garbled circuits and asks Alice to reveal them to him.
Bob can detect if Alice cheated in garbling any of the circuits.
If Bob finds out that Alice garbled the majority of the circuits correctly, then he evaluates the rest of them and chooses the most recurring output as the final output.
Bob selects the majority instead of terminating the protocol in case of different outputs because Alice can create a faulty circuit that fails given a particular input.
Thus, Bob's termination can reveal his input to Alice.

The cut-and-choose method creates another security challenge that is to make sure that the parties are using the same inputs for the multiple invocations of the garbled circuit.
To solve this problem, the authors propose to use a commitment scheme for input such that parties cannot change their input after the start of the protocol.
The authors later improve their method by reducing number of commitments at the expense of garbling more circuits to make the overall protocol more efficient \cite{lindell2012secure}.
They showed that garbling $s$ circuits and opening half of them result in the cheating probability of $2^{-0.311s}$.
For example, one requires garbling $s=128$ circuits to achieve $2^{-40}$ cheating probability.
Most recently, Lindell (2016) improved the cut-and-chose strategy even further in \cite{lindell2016fast}.
He showed applying some twists to the cut-and-chose strategy can achieve $2^{-s}$ probability error given $s$ circuits.
This means one requires garbling only $s=40$ circuits to achieve $2^{-40}$ cheating probability.

Nielsen et al. (2009) proposed LEGO, an alternative method for securing \acrshort{gc} in the presence of malicious adversary \cite{nielsen2009lego}.
The main difference between LEGO and the conventional cut-and-chose methods  \cite{lindell2007efficient, shen2011two, lindell2012secure, mohassel2013garbled, lindell2016fast} is that instead of garbling the entire circuit for multiple times, Alice garbles a large number of NAND gates in LEGO.
And instead of asking to reveal the entire circuit, Bob asks her to open half of the garbled NAND gates.
Bob then constructs a fault-tolerant circuit based on the original circuit using the remaining unrevealed NAND gates and evaluate the circuit.
Since in this thesis, we focus on generating a Boolean function with the minimum number of non-XOR gates, LEGO cannot directly enjoy from the improvement of \gls{tinygarble} synthesis similar to the conventional cut-and-chose methods that reveal the entire circuit.
However, the \acrshort{gc} synthesis methodology presented in \chap{chap:syn} may improve the number of NAND gates required to create the fault-tolerate circuit for LEGO.

\subsection{Discussion}\label{sec:related-malicious-dis}
Securing \acrshort{gc} using the cut-and-chose strategy incurs a multiplicative factor of $s$ (the statistical security parameter) to the cost of \acrshort{gc} compared to the \acrshort{hbc} model.
Thus, reducing the cost of \acrshort{gc} under \acrshort{hbc} (one of the objectives of this thesis) results in an immediate reduction in the cost of \acrshort{gc} under the malicious adversary as well.

Since in this thesis, we use the \acrshort{gc} protocol as a black-box, one can easily apply the cut-and-chose strategy presented in \cite{lindell2016fast} to secure our frameworks against malicious adversaries.
For instance, in the \gls{arm2gc} framework described in \sect{sec:processor-arm}, the Boolean circuit is the circuit of an \gls{arm} processor, and the inputs are parties' private inputs.
Both parties create and agree on a simplified circuit of the \gls{arm} processor that is going to be garbled/evaluated using the \gls{skipgate} algorithm.
Using Lindell's method, Alice and Bob first commit their inputs \cite{lindell2016fast}.
Then for achieving $2^{-40}$ cheating probability, Alice generates $40$ garbled circuits from the simplified \gls{arm} circuit and sends them to Bob.
Bob then asks Alice to reveal half of them to Bob, and if the majority of the garbled circuit are garbled correctly, he evaluates the rest and chooses the most recurring output as the final output.
Similarly, one can easily apply this to \gls{tinygarble} \acrshort{gc} engine to make it secure against the malicious adversary.

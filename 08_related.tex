% !TEX root = 0_main.tex
\chapter{Related Work}
We classify related work into compilers for \acrshort{gc}s (\sect{sect:GCCompiler}), libraries for \acrshort{gc}s (\sect{sect:GCLibs}), \acrshort{gc} implementations with hardware accelerators (\sect{sect:HWaccel}), and \acrshort{gc} implementations on mobile devices (\sect{sect:mobile}).

\section{Compiler for Garbled Circuits}
The following tools compile high level function descriptions into a Boolean circuit which can be used in \acrshort{gc}.
The first realization of \acrshort{gc}s was Fairplay \cite{malkhi2004fairplay} which provides a custom high level procedural language called \acrfull{sfdl} that is compiled into a circuit description language, \acrfull{shdl}.
Another compiler is TASTY \cite{henecka2010tasty} which allows to combine garbled circuits and homomorphic encryption.
The compiler of \cite{kreuter2012billion} for the first time showed scalability to circuits consisting of billions of gates, e.g., a 4095x4095-bit edit distance circuit with almost 6 Billion gates.
The compiler of \cite{franz2014cbmc} allows to use a subset of \acrshort{ansi} C as input language.

To reduce the memory overhead for storing large circuits and hence increase scalability, \gls{pcf} \cite{kreuter2013pcf} introduced loops that, if given manually in the high level language, are kept until the \acrshort{gc} evaluation.
In contrast to \gls{pcf}, \gls{tinygarble} allows to infer loops automatically and also allows to optimize across multiple sub-circuits.

\section{Libraries for Garbled Circuits}
Instead of compiling circuits, FastGC \cite{huang2011faster} proposed to use a library-based approach where circuits can be programmed and easily integrated into high-level applications.
Another \acrshort{gc} library is VMCrypt \cite{malka2011vmcrypt} that allows to dynamically construct and deconstruct sub-circuits.
FastGC was extended in \cite{henecka2013faster} to re-use the same sub-circuits.
Another library for secure computation is ABY that allows the efficient combination of multiple secure computation approaches \cite{demmler2015aby}.

In all these library-based approaches the circuits and their decomposition into sub-circuits has to be specified manually by the programmer, whereas we provide an automated approach.

\section{\acrshort{gc} Implementations with Hardware Accelerators}
The following works provide better performance by implementing garbled circuits in hardware, on \acrshort{gpu}s, or using \acrshort{aes-ni} available in recent \acrshort{cpu}s.
These works can benefit from the compact representation generated by \gls{tinygarble}.

J\"arvinen et al., \cite{jarvinen2010garbled} proposed a generic hardware architecture for \acrshort{gc}.
They realized two \acrshort{fpga}-based prototypes: a system-on-a-programmable-chip with access to a hardware crypto accelerator targeting smartcards and smartphones, and a stand-alone hardware implementation targeting \acrshort{asic}s.

Recently, several accelerations of \acrshort{gc}s using \acrshort{gpu}s have been proposed.
Husted et al., implemented Yao's \acrshort{gc} by using optimizations such as Free XOR, pipelining, and \acrshort{ot} extension \cite{husted2013gpu}.
Pu et al., realized dynamic programming based on \acrshort{gc} to solve the Edit-Distance and the Smith-Waterman problems \cite{pu2013computing}.
They also used the same optimizations as \cite{husted2013gpu} along with permute-and-encrypt, efficient lookup-table design, and compact circuits \cite{pu2013computing}.
Frederiksen et al., implemented a secure computation protocol with security against malicious adversaries based on cut-and-choose of Yao's garbled circuit and an efficient \acrshort{ot} extension for two-party computation on \acrshort{gpu}s \cite{frederiksen2013fast}.

Bellare et al., propose JustGarble in which they use fixed-key \acrshort{aes} for circuit-garbling \cite{bellare2013efficient}.
They show their implementation using \acrshort{aes-ni} can efficiently garble and execute a circuit far faster than any prior report.

\section{\acrshort{gc} Implementations on Mobile Devices}
Our approach for generating compact circuit representations is also beneficial when performing secure computation on resource constrained devices such as mobile devices which have a limited amount of main memory.
Secure computation on mobile devices using garbled circuits was proposed in \cite{huang2011privacy}.
Also the protocol described in \cite{demmler2014ad}, which uses a smartcard installed in the mobile device, can benefit from our more compact circuit representation.
In \cite{carter2016secure, carter2014whitewash}, the mobiles no longer need to process circuits any more as \acrshort{gc} generation and evaluation is outsourced to cloud servers.

\section{\acrshort{hdl} Synthesis for Other \acrshort{sfe} protocols}

\section{Securing \acrshort{gc} against Malicious Adversary}
Generic ways of modifying \acrshort{gc}-based protocols such that they achieve security against stronger malicious adversaries have been proposed, e.g., \cite{lindell2007efficient, lindell2012secure, nielsen2009lego}.

\section{Related work to \gls{arm2gc}}
The idea of designing a custom programming language to describe and efficiently compile functions for secure evaluation dates back to Fairplay, the first \acrshort{gc} compiler \cite{malkhi2004fairplay}.
Fiarplay introduces a custom language created specifically to describe functions, namely the \acrfull{sfdl}.
\acrshort{sfdl} was compiled to \acrfull{shdl}.
More powerful languages and compilers were later presented \cite{HKSSW10, kreuter2012billion, rastogi2014wysteria}.
Introduction of a custom programming language is neither user-friendly, nor versatile when compared with a conventional programming languages like C.

Another approach adopted in FastGC~\cite{huang2011faster, HS13}, VMCRYPT~\cite{malka2011vmcrypt}, and ABY~\cite{demmler15aby} for \acrshort{gc} circuit generation is to design a library containing \acrshort{gc} optimized sub-circuits in a general purpose high-level language like Java.
This method requires the user to have a thorough understanding of the circuit description of the secure function as the circuits and their decomposition into sub-circuits has to be specified manually.

The first \acrshort{gc} implementation supporting a general purpose language is presented in~\cite{holzer2012secure}, which supports ANSI-C.
However, it supports only a subset of ANSI-C that is not compatible with many important primitives and therefore, not compatible with legacy codes.
The main drawback of~\cite{holzer2012secure} is compile-time loop unrolling that makes it unscalable with input size.
To cope with this problem, \gls{pcf} compiler presented in~\cite{kreuter2013pcf} introduces loops that are specified manually within the code and not rolled out until the \acrshort{gc} evaluation.
This compiler supports a more general version of C language.
However, in~\cite{holzer2012secure} and~\cite{kreuter2013pcf}, the code had to be compiled with their custom compiler.
As a result, users cannot benefit from the optimizations provided by general purpose compilers.
Moreover, these compilers are less scrutinized and therefore more prone to bugs.
In contrast, the \gls{arm2gc} framework supports any general purpose \gls{arm} compiler and thus benefit from all the state-of-the-art optimizations, supports legacy codes, and is fully verified.

The TinyGarle framework~\cite{songhori2015tinygarble} allows a user to describe the function with a Hardware Description Language (HDL) like Verilog or VHDL.
It presents customized \acrshort{gc} optimized libraries which enable synthesis of the HDL code with standard logic synthesis tools, thus, benefiting from the standard hardware optimizations.
TinyGarble also suggests sequential circuits to \acrshort{gc} to solve the unscalablity issue.
Unlike~\cite{kreuter2013pcf}, it allows to infer loops automatically and to optimize across multiple sub-circuits.
TinyGarble also limits the programmer to a hardware level language which is less user friendly than a high level compiler.
Our work utilizes TinyGarble's methodology to generate the most optimized Boolean circuit for the \gls{arm} processor.
The big advantage of \gls{arm2gc} is that the function to be evaluated securely can be written in any programming language and compiled with any \gls{arm} compiler of choice.

The work in~\cite{wang2015secure} accepts a function as a MIPS machine code, which allows the programmer to describe the function in a language of her choice and compile with a standard compiler.
They design a MIPS emulator to securely execute the code.
To avoid emulating the large number of instructions supported by the MIPS machine, they perform a data independent static analysis before execution of the program to build a small instruction bank and ALU circuit tailored for each processor cycle.
In contrast, our approach performs this optimization with bit-precision instead of instruction-precision.
Moreover, this is done in the runtime while the circuit remains the same for each cycle.
To solve the problem of secure conditional branch, they propose to pad \texttt{nop} instruction to parallel branches so that their lengths become equal.
This way when the code exits either of the branches, it ends up in the same instruction and the process can continue with less cost.
However, this approach increases the cost for conditional branches.
To mitigate this problem, we propose to use \gls{arm} processor which supports conditional execution and can replace these branches with conditional instructions (see \sect{ssec:arm}).
In rare cases where the \gls{arm} compiler fails to replace the conditional branch, we adopted their approach in padding the parallel branches with \texttt{nop} instruction.
Overall, our evaluation shows that \gls{arm2gc} outperforms their MIPS framework, for example by 4 orders of magnitudes for Hamming distance function, mostly thanks to the \gls{skipgate} algorithm and its bit-precision optimization.

The recent work of~\cite{mood2016frigate} studies the state-of-the-art compilers for performing secure function evaluation using high-level languages like C or Java.
They show that the majority of such compilers are not thoroughly validated and they reported the observed flaws in six commonly used platforms.
As they discuss in their paper, there are serious limitations for formal verification and due to its impracticality, they limit their analysis to validation by testing.
This type of testing does not detect all possible flaws in the compilation process.

The authors in~\cite{mood2016frigate} also introduce Frigate, a new C-style language for \acrshort{sfe} and the corresponding compiler.
Whereas in our work, we utilize C language with standard \gls{arm} cross compiler.
Our work also support any programming language and corresponding compiler.
As of now, Frigate only supports three different types (\texttt{uint\_t} , \texttt{int\_t}, and \texttt{struct\_t}).
The user can add her own types but it requires a good understanding of the internal structure of the compiler.
Since these three types have a specific bit length, the final computation is not bit-level efficient.
For example for a 9-bit comparison, Frigate needs to do the comparison for a given bit length of \texttt{int\_t}.
On the contrary, the \gls{arm2gc} framework eliminates unnecessary gates and evaluates the circuit only up to the number of bits needed.
Frigate divides the program into different functions and creates the circuit by calling the corresponding functions and as a result prohibits the overall circuit optimization.
In contrast, our \gls{arm} circuit is optimized globally using state-of-the-art hardware synthesis techniques.
Therefore, our overall platform is based on very well-developed and debugged tools that have been used in industry for many years.
Also, if any new update becomes available for these tools, they can effortlessly be incorporated into our framework.

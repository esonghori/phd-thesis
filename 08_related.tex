% !TEX root = 0_main.tex
\chapter{Related Work}
We classify related work into compilers for \acrshort{gc}s (\sect{sect:GCCompiler}), libraries for \acrshort{gc}s (\sect{sect:GCLibs}), \acrshort{gc} implementations with hardware accelerators (\sect{sect:HWaccel}), and \acrshort{gc} implementations on mobile devices (\sect{sect:mobile}).

\section{Compiler for Garbled Circuits}
The following tools compile high level function descriptions into a Boolean circuit which can be used in \acrshort{gc}.
The first realization of \acrshort{gc}s was Fairplay \cite{malkhi2004fairplay} which provides a custom high level procedural language called \acrfull{sfdl} that is compiled into a circuit description language, \acrfull{shdl}.
Another compiler is TASTY \cite{henecka2010tasty} which allows to combine garbled circuits and homomorphic encryption.
The compiler of \cite{kreuter2012billion} for the first time showed scalability to circuits consisting of billions of gates, e.g., a 4095x4095-bit edit distance circuit with almost 6 Billion gates.
The compiler of \cite{franz2014cbmc} allows to use a subset of \acrshort{ansi} C as input language.

To reduce the memory overhead for storing large circuits and hence increase scalability, PCF \cite{kreuter2013pcf} introduced loops that, if given manually in the high level language, are kept until the \acrshort{gc} evaluation.
In contrast to PCF, TinyGarble allows to infer loops automatically and also allows to optimize across multiple sub-circuits.

\section{Libraries for Garbled Circuits}
Instead of compiling circuits, FastGC \cite{huang2011faster} proposed to use a library-based approach where circuits can be programmed and easily integrated into high-level applications.
Another \acrshort{gc} library is VMCrypt \cite{malka2011vmcrypt} that allows to dynamically construct and deconstruct sub-circuits.
FastGC was extended in \cite{henecka2013faster} to re-use the same sub-circuits.
Another library for secure computation is ABY that allows the efficient combination of multiple secure computation approaches \cite{demmler2015aby}.

In all these library-based approaches the circuits and their decomposition into sub-circuits has to be specified manually by the programmer, whereas we provide an automated approach.

\section{\acrshort{gc} Implementations with Hardware Accelerators}
The following works provide better performance by implementing garbled circuits in hardware, on \acrshort{gpu}s, or using \acrshort{aes-ni} available in recent \acrshort{cpu}s.
These works can benefit from the compact representation generated by TinyGarble.

J\"arvinen et al., \cite{jarvinen2010garbled} proposed a generic hardware architecture for \acrshort{gc}.
They realized two \acrshort{fpga}-based prototypes: a system-on-a-programmable-chip with access to a hardware crypto accelerator targeting smartcards and smartphones, and a stand-alone hardware implementation targeting \acrshort{asic}s.

Recently, several accelerations of \acrshort{gc}s using \acrshort{gpu}s have been proposed.
Husted et al., implemented Yao's \acrshort{gc} by using optimizations such as Free XOR, pipelining, and \acrshort{ot} extension \cite{husted2013gpu}.
Pu et al., realized dynamic programming based on \acrshort{gc} to solve the Edit-Distance and the Smith-Waterman problems \cite{pu2013computing}.
They also used the same optimizations as \cite{husted2013gpu} along with permute-and-encrypt, efficient lookup-table design, and compact circuits \cite{pu2013computing}.
Frederiksen et al., implemented a secure computation protocol with security against malicious adversaries based on cut-and-choose of Yao's garbled circuit and an efficient \acrshort{ot} extension for two-party computation on \acrshort{gpu}s \cite{frederiksen2013fast}.

Bellare et al., propose JustGarble in which they use fixed-key \acrshort{aes} for circuit-garbling \cite{bellare2013efficient}.
They show their implementation using \acrshort{aes-ni} can efficiently garble and execute a circuit far faster than any prior report.

\section{\acrshort{gc} Implementations on Mobile Devices}
Our approach for generating compact circuit representations is also beneficial when performing secure computation on resource constrained devices such as mobile devices which have a limited amount of main memory.
Secure computation on mobile devices using garbled circuits was proposed in \cite{huang2011privacy}.
Also the protocol described in \cite{demmler2014ad}, which uses a smartcard installed in the mobile device, can benefit from our more compact circuit representation.
In \cite{carter2016secure, carter2014whitewash}, the mobiles no longer need to process circuits any more as \acrshort{gc} generation and evaluation is outsourced to cloud servers.

\section{\acrshort{hdl} Synthesis for Other \acrshort{sfe} protocols}

\section{Securing \acrshort{gc} against Malicious Adversary}
Generic ways of modifying \acrshort{gc}-based protocols such that they achieve security against stronger malicious adversaries have been proposed, e.g., \cite{lindell2007efficient, lindell2012secure, nielsen2009lego}.

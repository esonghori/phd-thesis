% !TEX root = 0_main.tex
\chapter{Garbled Circuit Library}\label{chap:library}
In this section, we introduce \gls{tinygarble} library that consists of \acrfull{gc} optimized circuits for complex mathematical/logical operations that can be used as building blocks practical applications.
The \acrfull{sscd} and \gls{netlist} files of the operations in the library can be found in the \gls{tinygarble} repository\footnote{\url{https://github.com/esonghori/\gls{tinygarble}/blob/master/scd/netlists/v.tar.gz}}.

\section{Division and Remainder}
The \gls{tinygarble} library includes the circuit for integer division which takes two numbers as inputs and outputs quotient and remainder (modulus).
The library includes the circuits for 16-, 32-, and 64-bit integers.
We also add two additional circuits in the library that output only quotient and remainder respectively and have a fewer number of gates compared to the original circuit.

\section{Floating Point Operations}
The \gls{tinygarble} library includes an extensive set of operations for both IEEE-754 single- and double precision floating-point numbers.
It supports addition (fp\_add), subtraction (fp\_sub), division (fp\_div), and multiplication (fp\_mult) of two inputs.
We also include a comparison circuit (fp\_cmp) that outputs three bits: less than, greater than, equal signals.
The natural exponentiation circuit (fp\_exp) receives a floating-point input $a$ and computes $e^a$.
The logarithm circuit (fp\_log2) calculates the logarithm of the input in base 2.
Other floating-point operations include square (fp\_square) and square-root (fp\_sqrt).

\section{Encoder}
Encoder circuit converts a one-hot representation into a binary one.
One-hot representation has only one active bit (usually equal one).
Number $i$ is represented in one-hot by activating the $i^{\text{th}}$ bit.
Thus, the encoder circuit outputs the index of the active input bit of the one-hot representation.
Therefore, for $n$-bit one-hot input, the output $o$ is $\log(n)$ bit wide.
For example, if the $n=16$ and the $5^{\text{th}}$ bit is one, the output should be $o=0101$ (4-bit wide).
If none of the inputs is one, the encoder outputs zero.

We implement the encoder operation using a recursive structure.
An encoder for $n$-bit input is implemented using two smaller encoders for $n/2$-bit input.
The first half of the input ($0$ to $n/2-1$) is given to the first small encoder, and the rest ($n/2$ to $n-1$) is given to the second one.
One of these two small encoders that receives the inactive part of input outputs zero and the other one outputs the binary representation in its half.
Depending on which half was active, a bit is attached to the final output as the \acrfull{msb}.
In other words, the \acrshort{msb} of $o$ is set to one if the output of the second encoder is nonzero and otherwise zero.

\section{Argmax}
Given an array of numbers, Argmax circuit outputs the largest number along with its index in the array.
The current implementation supports integers, but it can easily be extended to support any type of inputs with appropriate comparison block.
The size of the array ($n$) and the number of bits ($b$) for each number are parameters of the circuit and can be set before compilation.
The combinational circuit of Argmax has $n-1$ comparison blocks, $n-1$ two-to-one $b$-bit-wide \acrfull{mux}s, and $n-1$ two-to-one $\log_2{n}$-bit-wide \acrshort{mux}s.
Its sequential circuit has one comparison blocks, and two \acrshort{mux}s for selecting the index and the largest number and two registers to store them.
The sequential circuit has to be evaluated for $n-1$ sequential cycles.

\section{CORDIC}
\acrfull{cordic} circuit computes hyperbolic and trigonometric functions.
It is an iterative algorithm that improves the accuracy of the result by (typically) one bit at each iteration.
We have implemented \acrshort{cordic} as a sequential circuit which performs one iteration at each sequential cycle.
It includes a lookup table, shift, addition, and subtraction operations.
\acrshort{cordic} circuit takes three inputs ($x_0$, $y_0$, and $z_0$) and outputs three values ($x_n$, $y_n$, and $z_n$).
The subscript $n$ denotes the final outputs after running $n$ iterations \acrshort{cordic}.
All of the inputs, outputs, and intermediate results are in fixed-point representation.
The number of integer bits and fractional bits are parameters that can be set before compilation.

\acrshort{cordic} has three operation modes:
(i) Circular, (ii) hyperbolic, and (iii) linear.
The circular mode can rotate an arbitrary vector by a given angle.
With specifically chosen inputs ($x_0=1$, $y_0=0$, and $z_0=\theta$) in circular mode, the circuit computes the trigonometric functions ($x_n=\cos(\theta)$, $y_n=\sin(\theta)$, and $z_n=0$).
The circuit computes two different functions without facing any more overhead compared to computing only one of them.
Similarly, the hyperbolic mode can compute hyperbolic functions.
Please note that in the hyperbolic mode iterations $3\times i+1$ need to be computed twice.
The linear mode can compute the multiplication of the inputs ($x_n=x_0, y_n=y_0z_0, z_n=0$).
Using \acrshort{cordic}'s outputs, we can create a few other non-linear functions as well, for example,
$\tan(\theta)=\frac{\sin(\theta)}{\cos(\theta)}$,
$\tanh(\theta)=\frac{\sinh(\theta)}{\cosh(\theta)}$,
$\exp(x)=\sinh (x) + \cosh (x)$, and
$Sigmoid(\theta)=\frac{1}{1+\cosh(\theta)-\sinh(\theta)}$.

\section{Evaluation}
\tab{table:div} reports the result of integer division and remainder functions for various bit-widths.
Div\_rem function calculates both the reminder and division in the same circuit.
A few of the previous work reported the number of non-XOR gates for 32-bit integer division only: 1,437 \cite{mood2016frigate}, 1,210 \cite{zahur2015obliv}, and 2,236 \cite{liu2015oblivm}.
As shown in the table, the number of non-XOR gates for our 32-bit division is 599.
It means 2 to 3.7 times improvement compared to the previous custom compilers.

\begin{table}
\center
\caption{The result of integer division and reminder functions.}\label{table:div}
\begin{tabular}{l||r||r|r}
	Function                  & \multicolumn{1}{c|}{Bit-width} & \multicolumn{1}{c|}{Non-XOR} & \multicolumn{1}{c}{Total gates} \\ \hline \hline
\multirow{3}{*}{Reminder} & 16        & 186     & 500         \\
                          & 32        & 631     & 1,777       \\
                          & 64        & 2,290   & 6,636       \\ \hline \hline
\multirow{3}{*}{Division} & 16        & 170     & 452         \\
                          & 32        & 599     & 1,681       \\
                          & 64        & 2,226   & 6,444       \\ \hline \hline
\multirow{3}{*}{Div\_rem} & 16        & 186     & 501         \\
                          & 32        & 631     & 1,778       \\
                          & 64        & 2,290   & 6,637
\end{tabular}
\end{table}

\tab{table:float} illustrates the number of non-XOR and total gates for the floating point operations.
The table reports the results of both single precision (float) and double precision (double).
Pullonen et al. (2015) are the only previous work that reported the result for the similar floating-point operations \cite{pullonen2015combining}.
They used the custom compiler of CBMC-GC \cite{franz2014cbmc} to compile software implementations of these floating-point operations.
As can be seen in the table, the circuits in \gls{tinygarble} library outperforms the ones in \cite{pullonen2015combining}.

\begin{table}
\center
\caption{The result of floating-point functions.}\label{table:float}
\begin{tabular}{l|l||rr||rr||rr}
\multirow{2}{*}{Function}   & \multicolumn{1}{c||}{\multirow{2}{*}{Precision}} & \multicolumn{2}{c||}{\begin{tabular}[c]{@{}c@{}}Previous work\\\cite{pullonen2015combining}\end{tabular}}                             & \multicolumn{2}{c||}{\begin{tabular}[c]{@{}c@{}}\gls{tinygarble} \\Library\end{tabular}}                                & \multicolumn{2}{c}{Comparison}                   \\ \cline{3-8}
                            & \multicolumn{1}{c||}{}                           & \multicolumn{1}{c}{Non-XOR} & \multicolumn{1}{c||}{Total gates} & \multicolumn{1}{c}{Non-XOR} & \multicolumn{1}{c||}{Total gates} & \multicolumn{1}{c}{GTD} & \multicolumn{1}{c}{MFE} \\ \hline \hline
\multirow{2}{*}{fp\_add}    & Single                                          & 5,671                       & 7,052                            & 936                         & 1,562                            & -83.5\%                 & 4.51                    \\
                            & Double                                          & 13,129                      & 15,882                           & 2,030                       & 3,533                            & -84.5\%                 & 4.50                    \\ \hline \hline
\multirow{2}{*}{fp\_sub}    & Single                                          & 5,671                       & 7,052                            & 936                         & 1,562                            & -83.5\%	                & 4.51                    \\
                            & Double                                          & 13,129                      & 15,882                           & 2,032                       & 3,537                            & -84.5\%                 & 4.49                    \\ \hline
\multirow{2}{*}{fp\_mult}   & Single                                          & 5,138                       & 7,701                            & 3,554                       & 4,839                            & -30.8\%                 & 1.59                    \\
                            & Double                                          & 13,104                      & 25,276                           & 17,019                      & 23,484                           & 29.9\%                  & 1.08                    \\ \hline \hline
\multirow{2}{*}{fp\_div}    & Single                                          & 12,851                      & 21,384                           & 3,810                       & 6,275                            & -70.4\%                 & 3.41                    \\
                            & Double                                          & 36,133                      & 73,684                           & 17,585                      & 29,130                           & -51.3\%                 & 2.53                    \\ \hline \hline
\multirow{2}{*}{fp\_cmp}    & Single                                          & -                           & -                                & 213                         & 239                              & -                       & -                       \\
                            & Double                                          & -                           & -                                & 435                         & 463                              & -                       & -                       \\ \hline \hline
\multirow{2}{*}{fp\_exp}    & Single                                          & -                           & -                                & 12,596                      & 16,250                           & -                       & -                       \\
                            & Double                                          & 393,807                     & 579,281                          & -                           & -                                & -                       & -                       \\ \hline \hline
\multirow{2}{*}{fp\_log2}   & Single                                          & -                           & -                                & 13,072                      & 16,727                           & -                       & -                       \\
                            & Double                                          & -                           & -                                & 16,355                      & 23,484                           & -                       & -                       \\ \hline \hline
\multirow{2}{*}{fp\_square} & Single                                          & -                           & -                                & 1,763                       & 2,391                            & -                       & -                       \\
                            & Double                                          & -                           & -                                & 8,461                       & 11,457                           & -                       & -                       \\ \hline \hline
\multirow{2}{*}{fp\_sqrt}   & Single                                          & 35,987                      & 66,003                           & 1,842                       & 3,360                            & -94.9\%                 & 19.64                   \\
                            & Double                                          & 85,975                      & 169,932                          & 8,636                       & 15,801                           & -90.0\%                 & 10.75
\end{tabular}

\end{table}

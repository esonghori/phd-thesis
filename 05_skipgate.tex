% !TEX root = 0_main.tex
\chapter{{SkipGate}}
\section{{SkipGate} Algorithm}
SkipGate is a novel algorithm on top of the GC protocol.
It removes garbling cost for a gate when its output can either be computed independently by Alice and Bob or has no effect on the final output.
The cost reduction is especially significant in a sequential circuit where the control path is independent of the private inputs.
Before presenting SkipGate, we introduce the following notations and definitions.

In a classic Boolean circuit, each wire $w$ carries a value ($x_w\in\{0, 1\}$), whereas in a garbled circuit, each wire carries a pair of labels ($X_w^{0}$ and $X_w^{1}$) on Alice's side and one label ($X_w \in \{X_w^{0}, X_w^{1}\}$) on Bob's.
If $X_w = X_w^{0}$, the actual Boolean value is 0 and if $X_w = X_w^{1}$, it is 1.
This means that the information is shared between two parties.
In our scheme, we combine these notions of Boolean and garbled circuits.
Each wire either carries a Boolean value known to both parties independently (\textit{public} wire) or it carries a (pair of) label(s) (\textit{secret} wire).
As a result, the scheme supports functions in the form of $c = f(a, b, p)$ where $p$ is a public input known to both parties.

\subsection{Motivational Example}
Assume a sequential circuit that has a 2-to-1 MUX whose inputs come from two sub-circuits f$_0$ and f$_1$.
If at a certain clock cycle, the select wire of the MUX is public, say 1, both parties know that the gates in the sub-circuit f$_0$ do not need to be garbled/evaluated since they have no effect on the final output.
The gates in the MUX itself act as wires and pass the output of f$_1$ to the MUX output, thus they do not need to be garbled/evaluated in that clock cycle either.
In the following subsection, we explain how the SkipGate algorithm identifies such gates to reduce the garbling cost in circuits with public wires.

Note that the Boolean value of a wire can change at every clock cycle in a sequential garbled circuit~\cite{songhori2015tinygarble}.
The wire can also alter between being secret and public.
The SkipGate algorithm is executed each clock cycle by both parties just before GC operation to either pre-compute gates or mark them for skipping.
Thus, SkipGate is different compared to offline circuit simplification methods such as the one introduced in~\cite{pinkas2009secure} which remove gates with known constant inputs.
The constant gates are already removed in our circuits by the conventional HDL synthesis tools.

\subsection{Gate Categories}\label{ssec:skipgate_example}
The SkipGate algorithm classifies the gates into four categories in terms of the parties' knowledge about their inputs:

\begin{enumerate}[label=\roman*]
  \item \textit{Gate with two public inputs.} In this case, the output is public.
  \item \textit{Gate with one public input.}
  	Depending on the gate type, the output becomes either public or secret.
  	For example, for an AND gate with 0 at one input, the output becomes 0.
  	This means that if the secret input is not connected to any other gate, the gate generating it can be skipped for garbling/evaluation.
  	If the public input is 1, then the AND gate acts as a wire and the output wire carries the label of the secret input.
  \item \textit{Gate with secret inputs that have identical (or swapped) labels.}
    This indicates that the two secret inputs have identical (or inverted) Boolean values.
    (We will explain shortly how Bob identifies the swapped case.)
    Depending on the gate type, the output becomes either public or secret.
    For example, the output of an XOR gate with two inverted inputs (either secret or public) is always 1 (public).
  	Similar to (ii), the gate generating the inputs, if not connected to any other gates, can be skipped for garbling/evaluation.
  \item \textit{Gate with secret unrelated inputs.}
  	The output is always secret.
  	They have to be garbled/evaluated conventionally according to the GC protocol.
\end{enumerate}

\subsection{Algorithm}\label{ssec:skipgate_algorithm}
In the SkipGate algorithm, the netlist's inputs are marked as secret or public by the parties.
An integer, \texttt{label\_fanout}, is associated with each gate and indicates the number of times the gate's output label is used as a circuits output or an input to other gates.
Initially, the \texttt{label\_fanout} is set to the gate fanout in the circuit.
Next, SkipGate computes the outputs of the gates in categories i-iii.
In the process, \texttt{label\_fanout} of a gate may decrease if its output is connected to a gate in those categories.
If \texttt{label\_fanout} reaches 0, the gate is marked for skipping which in turn decreases the \texttt{label\_fanout} of its input gates recursively.
Finally, the gates in category iv that have not been marked for skipping are garbled/evaluated.
\alg{skipgate_algorithm} shows SkipGate for both Alice and Bob.
The algorithm receives a topologically sorted netlist.
For each clock cycle, Alice first runs the algorithm, generates garbled tables, and sends them to Bob.
Bob then starts the algorithm after receiving the garbled tables.
This process can be pipelined.

Alice and Bob find and compute the gates that belong to categories (i, ii) in Phase 1 of \alg{skipgate_algorithm}.
\texttt{label\_fanout}s of the gates in these categories are set to zero.
For gates in category ii, SkipGate decreases the \texttt{label\_fanout} of the secret input's originating gates recursively by invoking \texttt{recursive\_reduction} (\alg{skipgate_reduction}).
In Phase 2, both parties perform the same task for category iii.
Alice also generates garbled tables for gates with non-zero \texttt{label\_fanout} in category iv.
By the end of Phase 2, due to the recursive nature of the fanout reduction, \texttt{label\_fanout} of some gates that have already been garbled may become 0.
Alice filters the garbled tables that have non-zero \texttt{label\_fanout} to be sent to Bob in Phase 3.
This reduces the communication cost between Alice and Bob which is the main bottleneck of the GC protocol~\cite{gueron2015fast}.

Bob evaluates garbled tables sent by Alice in Phase 2.
He assigns random labels for gates in category iv that are not marked for skip but were filtered by Alice in her Phase 3.
These gates were marked for skip in Alice's Phase 2 during computation of category iii.
Bob uses the random labels to keep track of identical (or swapped) input labels to compute the output of the gates in category iii.

\begin{algorithm}[t!]
\caption{SkipGate Algorithm.}\label{skipgate_algorithm}
\textbf{Inputs:} Topologically sorted \texttt{netlist} whose \texttt{label\_fanout} is initialized with the gate fanout. \texttt{user} which indicates if the code is run by Alice or Bob. If run by Bob, \texttt{garbled\_tables} queue from Alice.\\
\textbf{Outputs:} If Alice, \texttt{garbled\_tables} queue containing the garbled table and gate ID. If Bob, the final output labels.\\
\begin{algorithmic}[1]
\STATE{\bf{Phase 1}:}
\FOR{g in netlist}
	\IF{g.i0 is public and g.i1 is public \textit{//category i}}
		\STATE{g.o = g.public\_calculate()}
		\STATE{g.fanout $= 0$}
	\ELSIF{g.i0 is public or g.i1 is public \textit{//category ii}}
		\STATE{g.o = g.half\_public\_calculate()}
		\IF{g.o is public}
			\STATE{g.fanout $= 1$ \textit{//to be zero in recursive\_reduction}}
			\STATE{netlist.recursive\_reduction(g)}
		\ENDIF
	\ENDIF
\ENDFOR
\STATE{\bf{Phase 2}:}
\FOR{g in netlist where g.fanout $> 0$}
	\IF{(g.i0.label is equal g.i1.label or\\
		g.i0.label is inverted g.i1.label) \textit{//category iii}}
		\STATE{g.o = g.related\_secret\_calculate()}
		\IF{g.o is public}
			\STATE{g.fanout $= 1$ \textit{//to be zero in recursive\_reduction}}
			\STATE{netlist.recursive\_reduction(g)}
		\ENDIF
	\ELSE
    \STATE{\textit{//category iv}}
    \IF{user is Alice}
      \STATE{(g.o, g.table) = g.garble() \textit{//table=null for XOR}}
      \IF{g is non-XOR}
       \STATE{garbled\_tables.enqueue(g.id, g.table)}
      \ENDIF
    \ELSIF{user is Bob}
      \IF{g is XOR}
       \STATE{g.o = g.eval()}
      \ELSIF{g.id is garbled\_tables.top().id}
       \STATE{g.o = g.eval(garbled\_tables.dequeue().table)}
      \ELSE
       \STATE{g.o = random()}
      \ENDIF
    \ENDIF
	\ENDIF
\ENDFOR
\STATE{\bf{Phase 3}:}
\IF{user is Alice}
  \STATE{garbled\_tables.filter(t : netlist[t.id].fanout $> 0$)}
\ENDIF
\end{algorithmic}
\end{algorithm}

\alg{skipgate_reduction} illustrates the pseudo-code for the recursive fanout reduction.
It receives the netlist and a gate inside the netlist.
It first decreases the \texttt{label\_fanout} of the given gate.
If the \texttt{label\_fanout} becomes 0, it recursively calls itself with the gates that generate the secret input(s).

\begin{algorithm}[t!]
\caption{Recursive Fanout Reduction in SkipGate Algorithm.}
\textbf{Inputs:} \texttt{netlist} and \texttt{g} (to start the reduction).\\
\textbf{Outputs:} None.\\
\label{skipgate_reduction}
\begin{algorithmic}[1]

\STATE{netlist.recursive\_reduction(g):}
\IF{g.fanout is  0}
	\STATE{return}
\ENDIF
\STATE{g.fanout = g.fanout - 1}
\IF{g.fanout is  0}
	\IF{g.i0 is secret}
		\STATE{netlist.recursive\_reduction(netlist[g.i0])}
	\ENDIF
	\IF{g.i1 is secret}
		\STATE{netlist.recursive\_reduction(netlist[g.i1])}
	\ENDIF
\ENDIF
\end{algorithmic}
\end{algorithm}

\subsection{Identification of identical and inverted labels} According to the GC protocol, Bob receives only one label $X_w$ for each secret wire $w$.
Due to Free XOR~\cite{kolesnikov2008improved}, he does not need to do anything with a label when it passes a NOT gate because the label's association with Boolean value is flipped by Alice.
Thus, he cannot tell apart an identical and inverted secret values just by storing labels.
However, he can keep track of these flips by storing one bit along with the label.
After evaluating a NOT gate, he simply flips the bit.
The bit helps him to differentiate between identical and inverted secret values which is crucial for Phase 2.

\subsection{Security Proof} The GC protocol is secure under honest-but-curious adversary model for any function in the form of $f(a,b)$.
In our scheme, the function is in the form of $f(a,b,p)$.
SkipGate dynamically reduces $f(a,b,p)$ into $f_p(a,b)$ and garbles/evaluates $f_p$.
In this reduction process, we only use the public input $p$ which is known to both parties.
In other words, no information about the private inputs $a$ and $b$ is revealed.
Therefore, the security proof of the GC protocol still holds for SkipGate.
